\usepackage{tikz}
\usepackage{pgfplots}
\pgfplotsset{compat=1.18}
\usepackage{amsmath}
\usepackage{amssymb}
\usepackage{dsfont}
\usepackage{subcaption}
%\usepackage{txfonts}

\definecolor{bar}      {RGB}{238,170,169}
\definecolor{crook}    {RGB}{220,187,153}
\definecolor{crown}    {RGB}{204,204,136}
\definecolor{sphinx}   {RGB}{186,221,153}
\definecolor{snake}    {RGB}{169,238,170}
\definecolor{yacht}    {RGB}{153,221,187}
\definecolor{chevron}  {RGB}{136,204,204}
\definecolor{signpost} {RGB}{153,187,221}
\definecolor{lobster}  {RGB}{170,170,238}
\definecolor{hook}     {RGB}{186,153,221}
\definecolor{hexagon}  {RGB}{204,136,204}
\definecolor{butterfly}{RGB}{221,153,186}

% eisenstein integers are of the form a+b\omega where \omega = \frac{-1 + i\sqrt{3}}{2}
% -1/2 + \sqrt{3}/2 i

\newcommand{\set}[1]{\{#1\}}

\DeclareMathOperator{\rmd}{d}
\DeclareMathOperator{\abs}{abs}

\pgfmathparse{sqrt(3)/2}
\edef\sqthreeotwo{\pgfmathresult}

\def\eisToCar#1#2{
  \edef\a{#1}
  \edef\b{#2}
  \pgfmathparse{-\b/2}
  \edef\bwRe{\pgfmathresult}
  \pgfmathparse{\b*\sqthreeotwo}
  \edef\bwIm{\pgfmathresult}
  \pgfmathparse{\a+\bwRe}
  \xdef\currX{\pgfmathresult}
  \xdef\currY{\bwIm}%
}

%(0 -1) x   %% x -> -y
%(1  0) y   %% y -> x
\def\rotateNinety#1#2{ 
  \edef\x{#1}
  \edef\y{#2}
  \xdef\rotY{#1}
  \pgfmathparse{-#2}
  \xdef\rotX{\pgfmathresult}
  }

\gdef\one{1}
\gdef\zero{0}

\def\writeb#1{\pgfmathparse{#1 >= 0}\ifx\pgfmathresult\one+#1\else#1\fi}

% an eisenstein integer a+b\omega is manhattan distance d(a+b\omega) from 0
% d(a+b\omega) = abs(a) + abs(b) if ab <= 0
% d(a+b\omega) = max(abs(a), abs(b)) if ab > 0

\xdef\maxSteps{5}

\pgfmathparse{sqrt(3)}
\xdef\sqthree{\pgfmathresult}

\gdef\testInGrid#1#2#3{
  \pgfmathparse{#1*#2 <= 0}
  \edef\oppSign{\pgfmathresult}
  \ifx\oppSign\one
  \pgfmathparse{abs(#1) + abs(#2)}
  \else\pgfmathparse{max(abs(#1), abs(#2))}
  \fi
  \edef\manhattanDist{\pgfmathresult}
  \pgfmathparse{\manhattanDist <= #3}
  \xdef\withinGrid{\pgfmathresult}
}
